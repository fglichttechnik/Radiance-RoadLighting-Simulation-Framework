\documentclass[10pt,a4paper]{report}
\usepackage[utf8]{inputenc}
\usepackage[ngerman, english]{babel}
\usepackage{amsmath}
\usepackage{amsfonts}
\usepackage{amssymb}
\usepackage{graphicx}
\usepackage{float} % graphics with H option
\usepackage[left=2cm,right=2cm,top=2cm,bottom=2cm]{geometry}
\usepackage{pdflscape} 			% für querseite
\usepackage{longtable} 			% für seitenumbruch im table
%\usepackage{booktabs}			% toprule bottomrule im table
%	\newcommand{}{\specialrule{0.2em}{0em}{0em}}
%	\newcommand{\mymidrule}{\specialrule{0.1em}{0em}{0em}}
%	\newcommand{}{\specialrule{0.2em}{0em}{0em}} 
\usepackage{colortbl}			% use collor in table
\usepackage{color}				% define using colors
	\definecolor{dunkelgrau}{rgb}{0.8,0.8,0.8}
	\definecolor{hellgrau}{rgb}{0.95,0.95,0.95}
\usepackage[bookmarks,raiselinks,pageanchor,
            hyperindex,colorlinks,
            citecolor=black,linkcolor=black,
            urlcolor=blue,filecolor=black,
            menucolor=black]{hyperref} % using hypertext and urls

\author{Robert Franke}
\title{RoadRad Manual}

\begin{document}

\begin{titlepage}
\centering
\includegraphics[scale=0.09]{bilder/tub_logo}\\[3ex]
{\Large \textsc{Technische Universität Berlin}}\\[3ex]
{\Large Fakultät für Elektrotechnik und Informationstechnik}\\[3ex]
\vfill
{\Large \textbf{Manual}}\\[4ex]
{\large \textbf{RoadRad Documentation}}\\[5ex]
\vfill
\begin{tabular}{rl}
\hline\\
author :          		& \quad Robert Franke \\[1,5ex]

date :         			& \quad 26.\,11.\,2012\\[1,5ex]

by chair :				& \quad Lichttechnik\\[1,5ex]
                        & \quad Fakultät für Elektrotechnik und Informationstechnik\\[1,5ex]

\end{tabular}
\vfill
\end{titlepage}

\pagenumbering{roman}
\newpage\thispagestyle{empty}
\tableofcontents %Inhaltsverzeichnis
\newpage
\setcounter{page}{1}
\pagenumbering{arabic}

\chapter{Installation}
At first you need an UNIX operating system, for example Linux, Mac OS or cygwin  for MS Windows. Make sure, that your operating system has an updated python version ( eg. python 2.7, windows user get this with cygwin ). Then you have to install the python image library (PIL) (\url{http://www.pythonware.com/products/pil/}) and lxml (\url{http://www.lxml.de/}) depend on your python version and operating system. The PIL is for generating images and lxml for parsing the xml files ( windows user see additional information ).\\
\\
The second step is installing RADIANCE on your operating system. You get RADIANCE from \url{http://radsite.lbl.gov/radiance/download.html}. Run the makeall script and follow the installation description. Windows user must run this script from cygwin terminal, navigate to the unzipped radiance folder:
\begin{quote}
cd /hereIsMyDownloadedFolderWithMakeAllScript \\
./makeall install
\end{quote}

It will ask you, where you will install radiance. This is important for your path!
Use the following standard or you must configure you path after installation.
\begin{quote}
bin : /usr/local/ray/bin\\
lib : /usr/local/ray/lib
\end{quote}

Then add the following lines to .bashrc in your home directory (if not exist, add to .bash\_profile). ( You can use the bashrc ( docu/helbInstall ) file, which you get from RoadRad Repository )

\begin{quote}
RAY=/usr/local/ray\\
RAYBIN=/usr/local/ray/bin\\
RAYPATH=.:\$RAY/lib export RAYPATH\\
MANPATH=\$RAY/doc/man:\$MANPATH export MANPATH\\
PATH=\$RAYBIN:\$PATH export PATH
\end{quote}

To get the source code you require Git. Then connect to \url{//limaster.li.tu-berlin.de/server} and
mount this drive as \glqq z\grqq. Now you can clone the repository with the source code.

\begin{quote}
git clone \\
z:/Mitarbeiter/Code/Simulation/Radiance/Strassenbeleuchtung/Repository/ YourDir/YourRepositoryName
\end{quote}

Now you have something like \glqq YourDir/YourRepositoryName/RoadRad/src\grqq. The last step, make a folder named scene in \glqq YourDir/YourRepositoryName/RoadRad/\grqq.

\section{Additional Information Mac OS}
There is a workaraound, if the normal radiance installation does not work. On the limaster server is a compiled radiance version, copy it to the system folder.

\begin{quote}
sudo su\\
cp -rv /Volumes/server/... /usr/local/ray\\
cd /usr/local\\
chmod -R 777 ray
\end{quote}

For Mac OS you can use macports to install all required software.\\
install Git:
\begin{quote}
sudo port install git-core $+$svn$+$bash\_completion
\end{quote}
choose you Python version or install
\begin{quote}
port select --list python\\
sudo port select --set python python27
\end{quote}
install PIL for Python 2.7
\begin{quote}
sudo port install py27-pil
\end{quote}
install lxml for Python 2.7
\begin{quote}
sudo port install py27-lxml
\end{quote}

\section{Additional Information Cygwin MS Windows}
To use RADIANCE on MS Windows, you need an UNIX emulator eg. cygwin (\url{http://cygwin.com/install.html}). At first load and execute setup.exe. Then choose Install from Internet.

\begin{figure}[H]
\centering
\includegraphics[scale=0.5]{bilder/cygwin1.pdf} 
\caption{installation options}
\end{figure}

In the next two steps you can choose your installation folder and temporary download folder. Then choose the direct connection to server.

\begin{figure}[H]
\centering
\includegraphics[scale=0.5]{bilder/cygwin2.pdf} 
\caption{internet connection}
\end{figure}

Now you have to choose an download server from list. (Try to take a fast server eg. from germany university).

\begin{figure}[H]
\centering
\includegraphics[scale=0.5]{bilder/cygwin3.pdf} 
\caption{download site}
\end{figure}

At last step you can choose your packages. If you dont know, which packages you need, choose install all.

\begin{figure}[H]
\centering
\includegraphics[scale=0.3]{bilder/cygwin4.pdf} 
\caption{packages selection}
\end{figure}

Now you can install PIL and lxml. Download compiled versions from internet into a subfolder from cygwin (eg. cygwin/home/User/Downloads). Open terminal and navigate to your downloaded folders. Use lxml and PIL setup.py.

\begin{quote}
python setup.py install
\end{quote}

If you get an error you have to rebuild your memory on cygwin. Open /bin/ash.exe and type:

\begin{quote}
/bin/rebaseall
\end{quote}

Now repeat the installation script.

\chapter{XML Variables}
\label{chap:xml_variables}

There are some specific variables saved by an external xml-file (sceneDescription.xml), given in src-folder. These variables modify the scene and simulation parameters. The following table describes the parameters and valid values.

\begin{landscape}
\pagestyle{empty}
\renewcommand{\arraystretch}{1.5}
	\begin{longtable}{lllp{4cm}p{6cm}}
	
	\rowcolor{dunkelgrau}
		XML  Child & XML Parameters & Typ & Values & Description \\
	
		Description & Title & String & & title of the scene \\
	 	& Environment & String & Grass, City & grass only planar scene, city two boxes left and right side \\
	 	& FocalLength & Float & 0 - 100 & length of ray focus in m \\
	
	\rowcolor{hellgrau}
		Road & NumLanes & Integereger & $\geq$ 1 & number of  numlanes \\\rowcolor{hellgrau}
		 & NumPoleField & Integer & $\geq$ 1 & number of measurment fields \\\rowcolor{hellgrau}
		 & LaneWidth & Float & $>$ 0 & lane width in m \\\rowcolor{hellgrau}
		 & SidewalkWidth & Float & $\geq$ 0 & width of sidewalk in m \\\rowcolor{hellgrau}
		 & Surface & String & R1, R2, R3, R4, plastic, C1, C2, C2W3, C2W4, BRDF & typ of road surface / pavement \\\rowcolor{hellgrau}
	 	 & qZero & Float & 0 $\geq$ 1 & factor to change integrated $q_0$ in r-table \\
	
		Calculation & DIN13201 & String & on, off & switch for calculation, in the moment the value doesnot matter  \\
		 & VeilingLuminance & String & On, off & switch for veiling luminance visualization \\
		 & VeilingLuminanceMethod & String & RP800, DIN13201 & will calculate the veiling luminance for given standard \\
		 & HeadlightOption & String & withTarget, fixed & fixed - only fixed headlights for threshold increment calculation and veiling luminance, withTarget - for all headlights included fixedTargetDistance ( veiling luminance will be an array )\\
		 & TresholdLuminanceFactor & Float & & Used for findglare, to visual glare pics \\
		 & Age & Int & $>$ 0 & age of viewer, used for threshold increment calculation \\
	
	\rowcolor{hellgrau}
		ViewPoint & Distance & Float & & distance to measurementfield in m \\\rowcolor{hellgrau}
		 & Height & Float & & height of the observers eye in m \\\rowcolor{hellgrau}
		 & TargetDistanceMode & String & fixedViewPoint, fixedTargetDistance & observer goes with target or stays at position \\\rowcolor{hellgrau}
		 & XOffset & Float & & offset of the x-position of the observer in m \\\rowcolor{hellgrau}
		 & ViewDirection & String & first, last, fixedRP800 & viewing point \\
	
		Target & Size & Float & 0.3 & size of the target \\
		 & Reflectancy & Float & 0 $\geq$ 1 & absorption of the target \\
		 & Specularity & Float & 0 $\geq$ 1 & 0 \= 100 \% diffus, 1 \= 100 \% specular \\
		 & Roughness & Float & 0 $\geq$ 1 & only for plastic if its non linear \\
		 & Position & String & Left, Right, Center & x-position on numlane for target and observer \\ 
		 & OnLane & Integer & $\geq$ NumLanes & target position on numlane \\
	
	\rowcolor{hellgrau}
		LIDC & Name & String & & name of the LIDC .ies file \\\rowcolor{hellgrau}
		 & LightSource & String & LED, HPS .. & light source type for rgb color \\\rowcolor{hellgrau}
		 & LightLossFactor & Float & 0 $\geq$ 1 & LLF light loass factor of RP800 \\\rowcolor{hellgrau}
		 & SPRatio & Float & & Scotopic to Photopic Ration \\
	
		Headlight & LIDC & String & & name of the given LIDC declared in LIDC \\
		 & HeadlightDistanceMode & String & fixedHeadlightPosition, fixedTargetDistance & goes with target or stays at position \\
		 & Distance & Float & & distance of headlights in m \\
		 & Height & Float & 0.8 & height of the headlights in m \\ 
		 & Width & Float & 1.8 & width of the headlights in m \\
		 & SlopeAngle & Float & 0 - 90 & angle of slope in degree \\
		 & LightDirection & String & same, opposite & headlights in observer view direction or opposite \\
		 & OnLane & Integer & $\geq$ NumLanes & headlight position on numlane \\
	
	\rowcolor{hellgrau}
		PoleArray & Height & Float & & height of all poles in array must be same in m \\\rowcolor{hellgrau}
		 & Spacing & Float & & space between all single poles in m \\\rowcolor{hellgrau}
		 & Overhang & Float & & x-offset of the polearm in m \\\rowcolor{hellgrau}
		 & IsStaggered & String & True, False & first light begins in the middle of measurement field \\\rowcolor{hellgrau}
		 & LIDC & String & & name of the declacered LIDC in LIDC \\\rowcolor{hellgrau}
		 & Side & String & Left, Right & polearray left or right side of lanes \\
	
		PoleSingle & Height & Float & & height of pole in m \\
		 & Overhang & Float & & x-offset of the polearm in m \\
		 & LIDC & String & & name of the declacered LIDC in LIDC \\
		 & PositionY & Float & & y position of the single pole \\ 
		 & Side & String & Left, Right & pole left or right side of lanes \\
	
	\end{longtable} 

\end{landscape}

\chapter{Folder Structure}
\label{chap:folder_struc}

The RoadRad main programm has following important standard files and folders:
The src-folder is the folder for the working files. The following picture shows all files and subfolders.
\begin{figure}[H]
\includegraphics[scale=0.7]{bilder/src_folder.pdf} 
\caption{Structure of src folder.}
\end{figure}

All pavement surfaces are stored in the \glqq 3rdParty\grqq\ subfolder. If you are changing the names, you can not use them in the XML file. The following picture shows the structure and standard given surfaces.
\begin{figure}[H]
\includegraphics[scale=0.7]{bilder/3rdParty_folder.pdf} 
\caption{Structure of 3rdParty folder.}
\end{figure}

The last subfolder \glqq classes\grqq\ is very important. There are some fixed scene parameters added in the \textit{RoadScene.py}.
\begin{figure}[H]
\includegraphics[scale=0.7]{bilder/classes_folder.pdf} 
\caption{Structure of classes folder.}
\end{figure}


These fixed scene parameters are listed in the following table. You can change them by editing the following files in src-folder and subfolders.

\begin{landscape}
\pagestyle{empty}
\renewcommand{\arraystretch}{1.5}
	\begin{longtable}{lllp{4cm}p{6cm}}
	
	\rowcolor{dunkelgrau}
		Filename.py & Parameters & Typ & Values & Description \\
	
		ConfigGenerator & radDirSuffix & String & '/Rads' & folder for saving .rad and .cal files \\
	 	& lidcDirSuffix & String & '/LIDCs' &  folder for searching .ies files and saving .dat files \\
	 	& rTableDatSuffix & String & '/3rdParty/' &  folder for searching r-table files in specific subfolder \\
	 	& self.xmlConfigName & String & 'SceneDescription.xml' &  name of the given xml file, if you change, change in Simulator.py and Evaluator.py too! \\
	
	\rowcolor{hellgrau}
		RoadScene & sceneLength & Float & 8000 & length of road in m, important for ambient calculation \\\rowcolor{hellgrau}
		& sidewalkHeight & Float & 0.1 & height of sidewalk in m \\\rowcolor{hellgrau}
		& markingWidth & Float & 0.1 & width of dashline in m \\\rowcolor{hellgrau}
		& markingLength & Float & 0.1 & length of dashline in m \\\rowcolor{hellgrau}
		& poleRadius & Float & 0.05 & radius of pole cylinder in m \\\rowcolor{hellgrau}
		& numberOfLightsPerArray & Integer & 9 & maximum number of lights per array \\\rowcolor{hellgrau}
		& numberOfLightsBeforeMeasurementArea & Integer & 3 & maximum number of lights befor measuremnt field \\\rowcolor{hellgrau}
		& lidcRotation & Float & -90 & rotation of pole lidc in degree \\\rowcolor{hellgrau}
		& sensorHeight & Float & 8.9 & height of sensor in mm \\\rowcolor{hellgrau}
		& sensorWidth & Float & 6.64 & width of sensor in mm \\
	
		Simulator & horizontalRes & Float & 1380 & horizontal component of image resolution\\
		& verticalRes & Float & 1030 & vertical component of image resolution\\
		& makeFalsecolor & Boolean & True, False & switch to turn on or off falsecolor images\\
		& octDirSuffix & String & '/Octs' & folder suffixes for the .oct files\\
		& refOctDirSuffix & String & '/RefOcts' & folder suffixes for the refoct files\\
		& radDirSuffix & String & '/Rads' & folder suffixes for the .rad files\\
		& refPicDirSuffix & String & '/RefPics' & folder suffixes for refpic files\\
		& picDirSuffix & String & '/Pics' & folder suffixes for the pic files\\
		& lidcSuffix & String & '/LIDCs' & folder suffixes for the .ies files\\
		& picSubDirSuffix & String & '/pics' & subfolder suffixes for the .pic files\\
		& falsecolorSubDirSuffix & String & '/falsecolor' & subfolder suffixes for the falsecolor .pic files\\
		& pfSubDirSuffix & String & '/pfs' & subfolder suffixes for the .pf files\\
		& lumSubDirSuffix & String & '/lums' & subfolder suffixes for temp lum files\\
		& LMKSetMatFilename & String & '/LMKSetMat.xml' & name of the LMK Matlab XML file\\
	
	\rowcolor{hellgrau}
		Evaluator & octDirSuffix & String & '/Octs' &  folder suffix for the .oct files \\\rowcolor{hellgrau}
		& radDirSuffix & String & '/Rads' & folder suffix for the .rad files \\\rowcolor{hellgrau}
		& picDirSuffix & String & '/Pics' & folder suffix for the .pic files \\\rowcolor{hellgrau}
		& picSubDirSuffix & String & '/pics' & subfolder suffix for the pic files \\\rowcolor{hellgrau}
		& evalDirSuffix & String & '/Evaluation' & folder suffix for the calculation files \\\rowcolor{hellgrau}
		& horizontalRes & Float & 1380 & horizontal component of image resolution\\\rowcolor{hellgrau}
		& verticalRes & Float & 1030 & vertical component of image resolution\\
	
		Classes/ThresholdIncrement & octTempDirSuffix & String & '/TempOcts' & folder suffix for the temporary .oct files \\
    	& radDirSuffix & String & '/Rads' & folder suffix for the .rad files \\
    	& radTempDirSuffix & String & '/TempRads' & folder suffix for the temporary .rad files \\
    	& evalDirSuffix & String & '/Evaluation' & folder suffix for the calculation files \\
    	& lidcDirSuffix & String & '/LIDCs' & folder suffix for the .ies files \\
	
	\end{longtable} 
\end{landscape}

\chapter{Start Simulation}

To simulate your first scene, make sure, that you have the folder structure from chapter \ref{chap:folder_struc} and an execution folder named \glqq scene\grqq. In this folder you can put different subfolders with diffenrent titles, but each subfolder must have a \textit{SceneDescription.xml} and a \glqq LIDCs\grqq\ folder with ies-data, if you want to work with them.\\
Now you can configurate your scene by editing the \textit{SceneDescription.xml} in your subfolder like chapter \ref{chap:xml_variables}.

\begin{figure}[H]
\includegraphics[scale=0.45]{bilder/xml_struc2.pdf} 
\caption{Example of SceneDescription.xml}
\end{figure}

Now open you terminal and navigate to your src-folder. To start the simulation type in:

\begin{quote}
./RoadRad.py --dir YourSubfolder
\end{quote}

\begin{figure}[H]
\centering
\includegraphics[scale=0.5]{bilder/terminalStart.pdf} 
\caption{Example of SceneDescription.xml}
\end{figure}

If you want to simulate a lot of scenes, you can use the \textit{RoadRad\_batch.sh} in the src-folder. You can edit this file with a normal text editor. Add your folder names to the dirArray devided by space characters and add your working LMK database to the dirOfDatabase. Now you can start your simulation in terminal with:

\begin{quote}
./RoadRad\_batch.sh
\end{quote}

\begin{figure}[H]
\includegraphics[scale=0.45]{bilder/roadrad_batch.pdf} 
\caption{RoadRad\_batch.sh}
\end{figure}

\section{Ambient Simulation}
After all rads-data are made you have ten seconds time to choose between normal or ambient simulation. If you dont take a choice, a normal simulation will start. If you choose \glqq yes\grqq\ for ambient simulation, following \textit{rpict} paramters are used.
This values can be changed in the simulator.py.

\renewcommand{\arraystretch}{1.5}
	\begin{longtable}{p{3cm}p{3cm}p{3cm}p{3cm}p{3cm}}
		
		\rowcolor{dunkelgrau}
			ab & aa & ar & ad & as \\
		
			2 & .15 & 128 & 4096 & 1024 \\
		
	\end{longtable} 

\section{Evaluator}

\section{Video Simulator}
The video simulator is an optional addOn. You can activate the simulator from your terminal. Navigate to your src-folder and type in:

\begin{quote}
./RoadRad.py --dir YourSubfolder --video first value second value
\end{quote} 
The first value is the speed in kmh, the second value are the frames per second
\end{document}